\documentclass{standalone}

\usepackage{pgfplots}
\pgfplotsset{compat=1.15}
\usetikzlibrary{arrows,calc}
\usepackage{tkz-euclide}

\usepackage{graphics}

\pagestyle{empty}

\usepackage[greek,english]{babel}
\usepackage{textgreek}

\definecolor{AngleClr}{rgb}{0,0.39215686274509803,0}
\definecolor{ShapeClr}{rgb}{0.6,0.2,0}
\definecolor{BlueClr}{RGB}{5,81,163}
\definecolor{LinkClr}{HTML}{b20f0f}

\usepackage[hypertex,colorlinks=true,urlcolor=LinkClr]{hyperref}
\usepackage{../svg_labels}

% Reduce spacing after math block.
\usepackage{amssymb,amsmath,amsthm}
\setlength{\abovedisplayskip}{0pt}
\setlength{\belowdisplayskip}{0pt}

\begin{document}
\scalebox{1.5}{
\begin{tikzpicture}
\tkzSetUpLine[line width=1pt,color=black]
\tkzSetUpPoint[fill=black]

\tkzDefPoints{0/0/L,0.8/-3.7/M,5/0/K}

\tkzDefMidPoint(M,L) \tkzGetPoint{B}
\tkzDefMidPoint(L,K) \tkzGetPoint{A}
\tkzDefMidPoint(K,M) \tkzGetPoint{C}

\LS{ABCfill}{\tkzFillPolygon[fill=ShapeClr,fill opacity=0.1](A,C,B)}
\LS{ACBLfill}{\tkzFillPolygon[fill=ShapeClr,fill opacity=0.1](K,A,B,C)}
\LS{AKCBfill}{\tkzFillPolygon[fill=ShapeClr,fill opacity=0.1](A,L,B,C)}


\tkzDefTriangleCenter[ortho](A,C,B)\tkzGetPoint{H}

\tkzDefPointBy[projection=onto C--B](H)\tkzGetPoint{HA}
\tkzDefPointBy[projection=onto A--C](H)\tkzGetPoint{HB}
\tkzDefPointBy[projection=onto A--B](H)\tkzGetPoint{HC}

\LS{epsa}{\tkzDrawSegment[line width=0.5pt,color=black,dashed,dash pattern=on 1pt off 1.75pt,add=0.15 and 0.15](L,K)}
\LS{epsb}{\tkzDrawSegment[line width=0.5pt,color=black,dashed,dash pattern=on 1pt off 1.75pt,add=0.15 and 0.15](K,M)}
\LS{epsc}{\tkzDrawSegment[line width=0.5pt,color=black,dashed,dash pattern=on 1pt off 1.75pt,add=0.15 and 0.15](M,L)}


\LS{mua}{\tkzDrawSegment[line width=0.5pt,color=black,dashed,dash pattern=on 1pt off 1.75pt,add=0.2 and 0.2](A,HA)}
\LS{mub}{\tkzDrawSegment[line width=0.5pt,color=black,dashed,dash pattern=on 1pt off 1.75pt,add=0.2 and 0.2](B,HB)}
\LS{muc}{\tkzDrawSegment[line width=0.5pt,color=black,dashed,dash pattern=on 1pt off 1.75pt,add=0.2 and 0.2](C,HC)}

\LS{uaAngle}{\tkzMarkRightAngle[line width=0.5pt, size=.125,color=AngleClr,fill=AngleClr,fill opacity=0.1](H,HA,C)}
\LS{ubAngle}{\tkzMarkRightAngle[line width=0.5pt, size=.125,color=AngleClr,fill=AngleClr,fill opacity=0.1](H,HB,A)}
\LS{ucAngle}{\tkzMarkRightAngle[line width=0.5pt, size=.125,color=AngleClr,fill=AngleClr,fill opacity=0.1](H,HC,B)}

\LS{muaAngle}{\tkzMarkRightAngle[line width=0.5pt, size=.125,color=AngleClr,fill=AngleClr,fill opacity=0.1](H,A,L)}
\LS{mubAngle}{\tkzMarkRightAngle[line width=0.5pt, size=.125,color=AngleClr,fill=AngleClr,fill opacity=0.1](H,B,M)}
\LS{mucAngle}{\tkzMarkRightAngle[line width=0.5pt, size=.125,color=AngleClr,fill=AngleClr,fill opacity=0.1](H,C,K)}

\LS{ua}{\tkzDrawSegment[line width=0.5pt,color=black](A,HA)}
\LS{ub}{\tkzDrawSegment[line width=0.5pt,color=black](C,HC)}
\LS{uc}{\tkzDrawSegment[line width=0.5pt,color=black](B,HB)}



\LS{ABC}{
\tkzDrawPolygon[color=ShapeClr](A,B,C)
\tkzDrawPoints[size=3](A,B,C)
\tkzLabelPoint[above right](A){$\rm A$}
\tkzLabelPoint[below left](B){$\rm B$}
\tkzLabelPoint[below right](C){$\rm \Gamma$}
\tkzLabelSegment[below, scale=0.75](C,B){$\alpha$}
\tkzLabelSegment[right, scale=0.75](A,C){$\beta$}
\tkzLabelSegment[left, scale=0.75](A,B){$\gamma$}
}

\LS{KLM}{
\tkzDrawPoints[size=3](K,L,M)
\tkzLabelPoint[above right](K){$\rm K$}
\tkzLabelPoint[above left](L){$\rm \Lambda$}
\tkzLabelPoint[below](M){$\rm M$}}

\LS{H}{\tkzDrawPoints[size=3](H)}

\LS{ALlabel}{\tkzLabelSegment[above, scale=0.75](L,A){$\alpha$}}
\LS{AKlabel}{\tkzLabelSegment[above, scale=0.75](A,K){$\alpha$}}


\LS{OtherBetaGamma}{
\tkzLabelSegment[left, scale=0.75](M,B){$\beta$}
\tkzLabelSegment[left, scale=0.75](B,L){$\beta$}
\tkzLabelSegment[right, scale=0.75](M,C){$\gamma$}
\tkzLabelSegment[right, scale=0.75](C,K){$\gamma$}}



\node[text width=15cm, anchor=north west] at (6, 0) {
\LT{step1}{\textgreek{Έστω τρίγωνο $\rm AB\Gamma$.}}
\LT{step2}{\textgreek{Θα κατασκευάσουμε ένα τρίγωνο $\rm K\Lambda M$ ώστε οι \href{https://el.wikipedia.org/wiki/\%CE\%9C\%CE\%B5\%CF\%83\%CE\%BF\%CE\%BA\%CE\%AC\%CE\%B8\%CE\%B5\%CF\%84\%CE\%B7_\%CE\%B5\%CF\%85\%CE\%B8\%CF\%8D\%CE\%B3\%CF\%81\%CE\%B1\%CE\%BC\%CE\%BC\%CE\%BF\%CF\%85_\%CF\%84\%CE\%BC\%CE\%AE\%CE\%BC\%CE\%B1\%CF\%84\%CE\%BF\%CF\%82}{μεσοκάθετοι} των πλευρών του να συμπίπτουν με τα ύψη του $\rm AB\Gamma$.}}
\LT{step3}{\textgreek{Για τις μεσοκαθέτους γνωρίζουμε ότι διέρχονται από το ίδιο σημείο, το \href{https://el.wikipedia.org/wiki/\%CE\%A0\%CE\%B5\%CF\%81\%CE\%B9\%CE\%B3\%CE\%B5\%CE\%B3\%CF\%81\%CE\%B1\%CE\%BC\%CE\%BC\%CE\%AD\%CE\%BD\%CE\%BF\%CF\%82_\%CE\%BA\%CF\%8D\%CE\%BA\%CE\%BB\%CE\%BF\%CF\%82_\%CF\%84\%CF\%81\%CE\%B9\%CE\%B3\%CF\%8E\%CE\%BD\%CE\%BF\%CF\%85}{περίκεντρο}, και έτσι θα καταλήξουμε ότι και τα ύψη του $\rm AB\Gamma$ (ή οι προεκτάσεις τους) διέρχονται από το ίδιο σημείο.}}\vspace{0.2cm}

\LT{step4}{\textgreek{Θεωρούμε την ευθεία που διέρχεται από το $\rm A$ και είναι παράλληλη στο $\rm B\Gamma$,}} \LT{step5}{\textgreek{την ευθεία που διέρχεται από το $\rm B$ και είναι παράλληλη στο $\rm A\Gamma$ και την ευθεία που  διέρχεται από το $\rm \Gamma$ και είναι παράλληλη στο $\rm AB$.}} \LT{step6}{\textgreek{Έστω $\rm K\Lambda M$ το τρίγωνο που σχηματίζουν αυτές οι τρεις ευθείες.}}
\begin{itemize}\setlength\itemsep{0pt}
 \LT{step7}{\item \textgreek{Το τετράπλευρο $\rm KAB\Gamma$ είναι \href{https://el.wikipedia.org/wiki/\%CE\%A0\%CE\%B1\%CF\%81\%CE\%B1\%CE\%BB\%CE\%BB\%CE\%B7\%CE\%BB\%CF\%8C\%CE\%B3\%CF\%81\%CE\%B1\%CE\%BC\%CE\%BC\%CE\%BF}{παραλληλόγραμμο} καθώς οι πλευρές του είναι παράλληλες,}}\LT{step8}{\textgreek{ επομένως}
\[{\rm AK} = {\rm B\Gamma} = \alpha.\]}
\vspace{-0.7cm}
 \LT{step9}{\item \textgreek{Αντίστοιχα, από το παραλληλόγραμμο  $\rm A\Lambda \Gamma B$ έχουμε ότι}
\[{\rm \Lambda A} = {\rm B\Gamma} = \alpha.\]}
\vspace{-0.7cm}
 \LT{step10}{\item \textgreek{Συνεπώς το $\rm A$ είναι το μέσο του $\rm K\Lambda$ και η μεσοκάθετος του $\rm K\Lambda$ διέρχεται από το $\rm A$.}} \LT{step11}{\textgreek{Επίσης είναι κάθετη στο $\rm B\Gamma$ (καθώς ${\rm K\Lambda} \parallel {\rm B\Gamma}$), άρα το ύψος $\upsilon_{\alpha}$ ανήκει σε αυτή.}}
 \item \LT{step12}{\textgreek{Αντίστοιχα, τα $\rm B$ και $\rm \Gamma$ είναι τα μέσα των $\rm \Lambda M$ και $\rm KM$,}} \LT{step13}{\textgreek{και τα $\upsilon_{\beta}$ και $\upsilon_{\gamma}$ ανήκουν στις μεσοκαθέτους των $\rm \Lambda M$ και $\rm KM$.}}
\end{itemize}
\LT{step14}{\textgreek{Συνεπώς, καταλήγουμε ότι τα ύψη διέρχονται από το περίκεντρο του $\rm K\Lambda M$.}}
};

\end{tikzpicture}}
\end{document}
